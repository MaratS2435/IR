\documentclass[12pt,a4paper]{report}
\usepackage{graphicx} % Required for inserting images
\usepackage[T2A]{fontenc} % Кодировка шрифтов
\usepackage[utf8]{inputenc} % Кодировка исходного текста
\usepackage[english, russian]{babel} % Поддержка русского языка
\usepackage{csquotes}
\usepackage{textcomp}
\usepackage{fancyhdr}
\usepackage{amsmath,amssymb}
\usepackage{geometry}
\usepackage{listings}
\usepackage{xcolor}
\usepackage{hyperref}
\usepackage{booktabs}
\usepackage{longtable}
\usepackage{graphicx}
\usepackage{hyperref}
\usepackage{url}
\usepackage{graphicx}

\geometry{left=2cm,right=2cm,top=2cm,bottom=2cm}
\begin{document}

\begin{titlepage}
\begin{center}
\bfseries

{\Large Московский авиационный институт\\ (национальный исследовательский университет)}

\vspace{48pt}

{\large Факультет информационных технологий и прикладной математики}

\vspace{36pt}

{\large Кафедра вычислительной математики и~программирования}


\vspace{48pt}

Лабораторная работа \textnumero 1 по курсу \enquote{Информационный поиск}

\end{center}

\vspace{150pt}

\begin{flushright}
    \begin{tabular}{rl}
        Студент: & М.\,М. Сисенов \\
        Преподаватель: & А.\,А. Кухтичев \\
        Группа: & М8О-410Б \\
        Дата: & \\
        Оценка: & \\
        Подпись: & \\
    \end{tabular}
\end{flushright}

\vfill

\begin{center}
    \bfseries
    Москва, \the\year
\end{center}
\end{titlepage}

\pagebreak

\section*{Лабораторная работа \textnumero 1 \enquote{Добыча корпуса документов}} 

Необходимо подготовить корпус документов, который будет использован при выполнении остальных лабораторных работ:
\begin{itemize}
    \item Скачать его к себе на компьютер. В отчёте нужно указать источник данных.
    \item Ознакомиться с ним, изучить его характеристики. Из чего состоит текст? Есть ли
дополнительная мета-информация? Если разметка текста, какая она?
    \item Разбить на документы.
    \item Выделить текст.
    \item Найти существующие поисковики, которые уже можно использовать для поиска по
выбранному набору документов (встроенный поиск Википедии, поиск Google с использованием ограничений на URL или на сайт). Если такого поиска найти невозможно, то использовать корпус для выполнения лабораторных работ нельзя!
    \item Привести несколько примеров запросов к существующим поисковикам, указать недостатки в полученной поисковой выдаче.
\end{itemize}

В результатах работы должна быть указаны статистическая информация о корпусе:
\begin{itemize}
    \item Размер \enquote{сырых} данных.
    \item Количество документов.
    \item Размер текста, выделенного из \enquote{сырых} данных.
    \item Средний размер документа, средний объём текста в документе.
\end{itemize}

\pagebreak

\section*{Описание}

Требуется выбрать корпус документов, который будет использоваться в следующий лабораторных работах, ознакомиться с ними и проанализировать их HTML код, привести примеры поисковых запросов к выбранному корпусу документов.

\section*{Источник данных}
Были выбраны 2 сайта, главной тематикой которых являются статьи связанные с психологией:
\begin{itemize}
    \item \textbf{b17.ru} \url{https://www.b17.ru/} — сайт с огромным количеством статей и возможностью общения с профессиональными психологами.
    \item \textbf{psychologies.ru} \url{https://psychologies.ru/} — сайт имеет более популярный и менее научный формат, акцентирующий внимание на актуальных новостях и трендах.
\end{itemize}

\section*{Описание корпуса документов}
Причины выбора \textit{b17.ru} и \textit{psychologies.ru} :
\begin{itemize}
     \item \textit{Много текста}: На этих сайтах публикуются полноценные длинные статьи, а не короткие заметки. Это дает хороший объем данных, который необходим для качественной проверки закона Ципфа и работы стемминга.
    \item \textit{Встроенный поиск}: Сайты имеют внутреннюю поисковую систему, что может облегчить сравнение с внешними поисковиками.
    \item \textit{Простая верстка}: Структура сайтов понятна и логична (обычный HTML). Заголовки и тексты статей легко вытащить программно, не прибегая к сложным инструментам для обхода защиты или обработки скриптов.
\end{itemize}

\section*{Предварительный анализ структуры документов}
Каждая статья на сайтах представляет собой отдельный HTML-документ. По предварительному анализу можно выделить общие структурные элементы:
\begin{itemize}
    \item \textit{Заголовок}: Обычно размещается в теге \texttt{<h1>}.
    \item \textit{Основной текст}: Содержимое разбито на абзацы (\texttt{<p>}) и смысловые блоки. Часто используется микроразметка (например, атрибут \texttt{itemprop="articleBody"} или \texttt{class="article\_\_block article\_\_block\_type-text"}).
    \item \textit{Разметка}: Страницы используют современные семантические теги HTML5 (например, \texttt{<article>}, \texttt{<section>}), но также содержат большое количество служебных элементов (меню, реклама, ссылки), которые требуют фильтрации при парсинге.
\end{itemize}

\section*{Примеры документов}
Пример документа с \textbf{b17.ru}:

\begin{itemize}
    \item \textit{Размер сырого HTML}: 240 Kb
    \item \textit{Извлеченый текст}: 22 Kb
    \item \textit{Структура}: Документ имеет простую структуру: заголовок (<h1>), основной текст (itemprop="articleBody").
\end{itemize}
Средний результат документов с \textbf{psychologies.ru}:
\begin{itemize}
    \item \textit{Размер сырого HTML}: 235 Kb
    \item \textit{Извлеченый текст}: 14 Kb
    \item \textit{Структура}: Структура документа сложнее чем у b17, потому что сайт предлагает авторам больше возможностей для оформления (квизы, цитаты, картинки). Это заставляет более тщательно продумывать логику парсинга.
\end{itemize}

\section*{Поисковые запросы и анализ выдачи}
Для анализа были использованы Google и Яндекс. Чтобы задать конкретные ресурсы в поиске был использован оператор \textit{site:}.\\
Google - запрос к сайту b17.ru, запрос к сайту psychologies.ru, запрос к обоим сайтам: \\ \\
\includegraphics[width=80mm,height=50mm]{google_1.png} \includegraphics[width=80mm,height=50mm]{google_2.png} \\
\includegraphics[width=80mm,height=50mm]{google_3.png}

Аналогичные запросы с помощью Яндекса: \\ \\
\includegraphics[width=90mm,height=60mm]{yandex_1.png} \includegraphics[width=90mm,height=60mm]{yandex_2.png} \\
\includegraphics[width=90mm,height=60mm]{yandex_3.png} \\ 

Оба поисковика выдали примерно те же результаты, которые бы с высокой вероятностью соответствовали ожиданиям пользователя.

\section*{Вывод}

В ходе выполнения лабораторной работы был собран и проанализирован корпус документов на основе психологических порталов \textbf{b17.ru} и \textbf{psychologies.ru}. Была изучена структура HTML-страниц статей, выделены ключевые смысловые блоки (заголовок, основной текст). Определено, что для полноценного анализа необходимо не просто извлекать весь текст, а научиться выделять эти структурированные блоки отдельно. \\
Подготовленный корпус документов является релевантным, тематически однородным и достаточно объемным для выполнения последующих лабораторных работ по информационному поиску, таких как токенизация, стемминг, проверка закона Ципфа и построение булева поиска.

\begin{thebibliography}{99}
\bibitem{Kormen}
Маннинг, Рагхаван, Шютце
{\itshape Введение в информационный поиск} --- Издательский дом \enquote{Вильямс}, 2011. Перевод с английского: доктор физ.-мат. наук Д.\,А.\, Клюшина --- 528 с. (ISBN 978-5-8459-1623-4 (рус.))

\bibitem{b17} b17.ru: Сайт психологов №1\url {https://b17.ru}\\

\bibitem{psychologies} psychologies.ru: Онлайн-журнал про психологию \\  \url{https://psychologies.ru/}.

\end{thebibliography}
\pagebreak

\end{document}